%\extrapart{Bibliography and references}

% Pitman's exceptional situations paper can't be referenced, since it's only
% a working paper.

\todo{
This is just a personal remark on your question on the RRRS:
The language CUCH (Curry-Church) was implemented by 1964 and 
is a practical version of the lambda-calculus (call-by-name).
One reference you may find in Formal Language Description Languages
for Computer Programming T.~B.~Steele, 1965 (or so). -- Matthias Felleisen
}


\begin{thebibliography}{99}

\bibitem{SICP}
Harold Abelson and Gerald Jay Sussman with Julie Sussman.
{\em Structure and Interpretation of Computer Programs.}
MIT Press, Cambridge, 1985.

\bibitem{Bartley86}
David H.~Bartley and John C.~Jensen.
The implementation of PC Scheme.
In {\em Proceedings of the 1986 ACM Conference on Lisp
  and Functional Programming}, pages 86--93.

\bibitem{Scheme81}
John Batali, Edmund Goodhue, Chris Hanson, Howie Shrobe, Richard
  M.~Stallman, and Gerald Jay Sussman.
The Scheme-81 architecture---system and chip.
In {\em Proceedings, Conference on Advanced Research in VLSI},
  pages 69--77.
Paul Penfield, Jr., editor.
Artech House, 610 Washington Street, Dedham MA, 1982.

\todo{Church??}

\bibitem{RRRS}
William Clinger, editor.
The revised revised report on Scheme, or an uncommon Lisp.
MIT Artificial Intelligence Memo 848, August 1985.
Also published as Computer Science Department Technical Report 174,
  Indiana University, June 1985.

\bibitem{Clinger84}
William Clinger.
The Scheme 311 compiler: An exercise in denotational semantics.
In {\em Conference Record of the 1984 ACM Symposium on Lisp and
  Functional Programming}, pages 356--364.

\bibitem{Dybvig86}
R.~Kent Dybvig, Daniel P.~Friedman, and Christopher T.~Haynes.
Expansion-passing style: Beyond conventional macros.
In {\em Proceedings of the 1986 ACM Conference on Lisp and
  Functional Programming}, pages 143--150.

\bibitem{Eisenberg85}
Michael A.~Eisenberg.
Bochser: an integrated Scheme programming system.
MIT Laboratory for Computer Science Technical Report 349,
  October 1985.

\bibitem{Feeley86}
Marc Feeley.
Deux approches \`{a} l'implantation du language Scheme.
M.Sc.~thesis, D\'{e}partement d'Informatique et de Recherche
  Op\'{e}rationelle, University of Montreal, May 1986.

\bibitem{Felleisen86}
Matthias Felleisen, Daniel P.~Friedman, Eugene Kohlbecker, and Bruce Duba.
Reasoning with continuations.
In {\em Proceedings of the Symposium on Logic in Computer Science},
  pages 131--141.
IEEE Computer Society Press, Washigton DC, 1986.

\bibitem{Scheme311}
Carol Fessenden, William Clinger, Daniel P.~Friedman, and Christopher Haynes.
Scheme 311 version 4 reference manual.
Indiana University Computer Science Technical Report 137, February 1983.
Superceded by~\cite{Scheme84}.

\bibitem{Lisper}
Daniel P.~Friedman and Matthias Felleisen.
{\em The Little LISPer.}
Science Research Associates, second edition 1986.

\bibitem{Friedman84}
Daniel P.~Friedman, Christopher T.~Haynes, and Eugene Kohlbecker.
Programming with continuations.
In {\em Program Transformation and Programming Environments,\/}
  pages 263--274.
P.~Pepper, editor.
Springer-Verlag, 1984.

\bibitem{Scheme84}
D.~Friedman, C.~Haynes, E.~Kohlbecker, and M.~Wand.
Scheme 84 interim reference manual.
Indiana University Computer Science Technical Report 153, January 1985.

\bibitem{Friedman85}
Daniel P.~Friedman and Christopher T.~Haynes.
Constraining control.
In {\em Proceedings of the Twelfth Annual Symposium on Principles of
  Programming Languages}, pages 245--254.
ACM, January 1985.

\bibitem{Haynes84}
Christopher T.~Haynes, Daniel P.~Friedman, and Mitchell Wand.
Continuations and coroutines.
In {\em Conference Record of the 1984 ACM Symposium on Lisp and
  Functional Programming,} pages 293--298.

\bibitem{Haynes86}
Christopher T.~Haynes.
Logic continuations. 
In {\em Proceedings of the Third International Conference on
  Logic Programming,\/} pages 671--685.
Springer-Verlag, July 1986.
% and to appear in {\it The Journal of Logic Programming.}

\bibitem{Engines}
Christopher T.~Haynes and Daniel P.~Friedman.
Engines build process abstracions.
In {\em Conference Record of the 1984 ACM Symposium on Lisp and
  Functional Programming,\/} pages 18--24.

\bibitem{Henderson82}
Peter Henderson.  Functional Geometry.
In {\em Conference Record of the 1982 ACM Symposium on Lisp and
  Functional Programming}, pages 179--187.

\bibitem{Kohlbecker86}
Eugene Edmund Kohlbecker~Jr.
{\em Syntactic Extensions in the Programming Language Lisp.}
PhD thesis, Indiana University, August 1986.

\bibitem{Kranz86}
David Kranz, Richard Kelsey, Jonathan Rees, Paul Hudak, James Philbin,
  and Norman Adams.
Orbit: An optimizing compiler for Scheme.
In {\em Proceedings of the SIGPLAN '86 Symposium on Compiler
  Construction}, pages 219--233.
ACM, June 1986.
Proceedings published as {\em SIGPLAN Notices} 21(7), July 1986.

\bibitem{Landin65}
Peter Landin.
A correspondence between Algol 60 and Church's lambda notation: Part I.
{\em Communications of the ACM} 8(2):89--101, February 1965.

\bibitem{McDermott80}
Drew McDermott.
An efficient environment allocation scheme in an interpreter for a
  lexically-scoped lisp.
In {\em Conference Record of the 1980 Lisp Conference,} pages
  154--162.
The Lisp Conference, P.O.~Box 487, Redwood Estates CA,
  1980.
Proceedings reprinted by ACM.

\bibitem{MITScheme}
MIT Department of Electrical Engineering and Computer Science.
Scheme manual, seventh edition.
September 1984.

\bibitem{Muchnick80}
Steven S.~Muchnick and Uwe F.~Pleban.
A semantic comparison of Lisp and Scheme.
In {\em Conference Record of the 1980 Lisp Conference}, pages 56--64.
The Lisp Conference, 1980.

\bibitem{Naur63}
Peter Naur et al.
Revised report on the algorithmic language Algol 60.
{\em Communications of the ACM} 6(1):1--17, January 1963.

\bibitem{Penfield81}
Paul Penfield, Jr.
Principal values and branch cuts in complex APL.
In {\em APL '81 Conference Proceedings,} pages 248--256.
ACM SIGAPL, San Francisco, September 1981.
Proceedings published as {\em APL Quote Quad} 12(1), ACM, September 1981.

\bibitem{Pitman85}
Kent M.~Pitman.
Exceptional situations in Lisp.
MIT Artificial Intelligence Laboratory Working Paper 268, February 1985.

\bibitem{Pitman83}
Kent M.~Pitman.
The revised MacLisp manual (saturday evening edition).
MIT Artificial Intelligence Laboratory Technical Report 295, May 1983.

\bibitem{Pitman80}
Kent M.~Pitman.
Special forms in Lisp.
In {\em Conference Record of the 1980 Lisp Conference}, pages 179--187.
The Lisp Conference, 1980.

\bibitem{Rees82}
Jonathan A.~Rees and Norman I.~Adams IV.
T: A dialect of Lisp or, lambda: The ultimate software tool.
In {\em Conference Record of the 1982 ACM Symposium on Lisp and
  Functional Programming}, pages 114--122.

\bibitem{Rees84}
Jonathan A.~Rees, Norman I.~Adams IV, and James R.~Meehan.
The T manual, fourth edition.
Yale University Computer Science Department, January 1984.

\bibitem{Reynolds72}
John Reynolds.
Definitional interpreters for higher order programming languages.
In {\em ACM Conference Proceedings}, pages 717--740.
ACM, \todo{month?}~1972.

\bibitem{Rozas84}
Guillermo J.~Rozas.
Liar, an Algol-like compiler for Scheme.
S.~B.~thesis, MIT Department of Electrical Engineering and Computer
  Science, January 1984.

\bibitem{Smith84}
Brian C.~Smith.
Reflection and semantics in a procedural language.
MIT Laboratory for Computer Science Technical Report 272, January 1982.

\bibitem{Srivastava85}
Amitabh Srivastava, Don Oxley, and Aditya Srivastava.
An(other) integration of logic and functional programming.
In {\em Proceedings of the Symposium on Logic Programming}, pages 254--260. 
IEEE, 1985.

\bibitem{Stallman80}
Richard M.~Stallman.
Phantom stacks---if you look too hard, they aren't there.
MIT Artificial Intelligence Memo 556, July 1980.

\bibitem{Imperative}
Guy Lewis Steele Jr.~and Gerald Jay Sussman.
Lambda, the ultimate imperative.
MIT Artificial Intelligence Memo 353, March 1976.

\bibitem{Declarative}
Guy Lewis Steele Jr.
Lambda, the ultimate declarative.
MIT Artificial Intelligence Memo 379, November 1976.

\bibitem{Debunking}
Guy Lewis Steele Jr.
Debunking the ``expensive procedure call'' myth, or procedure call
  implementations considered harmful, or lambda, the ultimate GOTO.
In {\em ACM Conference Proceedings}, pages 153--162.
ACM, 1977.

\bibitem{Macaroni}
Guy Lewis Steele Jr.
Macaroni is better than spaghetti.
In {\em Proceedings of the Symposium on Artificial Intelligence and
  Programming Languages}, pages 60--66.
These proceedings were published as a special joint issue of {\em
  SIGPLAN Notices} 12(8) and {\em SIGART Newsletter} 64, August 1977.

\bibitem{Rabbit}
Guy Lewis Steele Jr.
Rabbit: a compiler for Scheme.
MIT Artificial Intelligence Laboratory Technical Report 474, May 1978.

\bibitem{CLoverview}
Guy Lewis Steele Jr.
An overview of Common Lisp.
In {\em Conference Record of the 1982 ACM Symposium on Lisp and
  Functional Programming}, pages 98--107.

\bibitem{CLtL}
Guy Lewis Steele, Jr.
{\em Common Lisp: The Language.}
Digital Press, Burlington MA, 1984.

\bibitem{Scheme78}
Guy Lewis Steele, Jr.~and Gerald Jay Sussman.
The revised report on Scheme, a dialect of Lisp.
MIT Artificial Intelligence Memo 452, January 1978.

\bibitem{TAOTI}
Guy Lewis Steele, Jr.~and Gerald Jay Sussman.
The art of the interpreter, or the modularity complex (parts zero, one,
  and two).
MIT Artificial Intelligence Memo 453, May 1978.

\bibitem{DOALBP}
Guy Lewis Steele, Jr.~and Gerald Jay Sussman.
Design of a Lisp-based processor.
{\em Communications of the ACM} 23(11):628--645, November 1980.

\bibitem{Dream}
Guy Lewis Steele, Jr.~and Gerald Jay Sussman.
The dream of a lifetime: a lazy variable extent mechanism.
In {\em Conference Record of the 1980 Lisp Conference}, pages 163--172.
The Lisp Conference, 1980.

\bibitem{Scheme75}
Gerald Jay Sussman and Guy Lewis Steele, Jr.
Scheme: an interpreter for extended lambda calculus.
MIT Artificial Intelligence Memo 349, December 1975.

\bibitem{Scheme79}
Gerald Jay Sussman, Jack Holloway, Guy Lewis Steele, Jr., and Alan Bell.
Scheme-79---Lisp on a chip.
{\em IEEE Computer} 14(7):10--21, July 1981.

\bibitem{Stoy77}
Joseph E.~Stoy.
{\em Denotational Semantics: The Scott-Strachey Approach to
  Programming Language Theory.}
MIT Press, Cambridge, 1977.

\bibitem{TI85}
Texas Instruments, Inc.
{\em TI Scheme Language Reference Manual.}
Preliminary version 1.0, November 1985.

\bibitem{Wand78}
Mitchell Wand.
Continuation-based program transformation strategies.
{\em Journal of the ACM} 27(1):174--180, 1978.

\bibitem{Wand80}
Mitchell Wand.
Continuation-based multiprocessing.
In {\em Conference Re\-cord of the 1980 Lisp Conference}, pages 19--28.
The Lisp Conference, 1980.

\end{thebibliography}
